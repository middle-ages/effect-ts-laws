\usepackage[vcentering,dvips]{geometry}
\geometry{papersize={26in,22in},total={22in,18in}}

\usepackage{lmodern}
\usepackage[T1]{fontenc}
\usepackage[utf8]{inputenc}
\usepackage{array}
\usepackage{hyperref}
\usepackage[dvipsnames]{xcolor}
\usepackage{amsfonts}
\usepackage{amsmath}
\usepackage{amssymb}
\usepackage{amstext}
\usepackage{amsthm}
\usepackage{array}
\usepackage{bold-extra}
\usepackage{booktabs}
\usepackage{graphicx}
\usepackage{mathtools}
\usepackage{multicol}
\usepackage{multirow}
\usepackage{nicefrac, xfrac}
\usepackage{relsize}
\usepackage{standalone}
\usepackage{trimclip}

\pagestyle{empty}

\hypersetup{colorlinks=true,linkcolor=blue,urlcolor=blue}

\definecolor{ordGray}{gray}{0.4}
\definecolor{sigGray}{gray}{0.35}

\newcolumntype{K}{>{\(}l<{\)}}

\renewcommand*\ttdefault{cmvtt}

\DeclareFontFamily{U}{mathb}{\hyphenchar\font45}
\DeclareFontShape{U}{mathb}{m}{n}{<5> <6> <7> <8> <9> <10> gen * mathb <10.95> mathb10 <12> <14.4> <17.28> <20.74> <24.88> mathb12 }{}
\DeclareSymbolFont{mathb}{U}{mathb}{m}{n}
\DeclareFontSubstitution{U}{mathb}{m}{n}
\DeclareMathSymbol{\sqsubsetneq}{1}{mathb}{"88}

\newcommand{\Link}[2]{\href{https://middle-ages.github.io/effect-ts-laws-docs/#1}{#2}}

% • Kern
\newcommand{\x}{\kern-0.5pt}
\newcommand{\X}{\kern-1pt}
\newcommand{\Xa}{\kern-1.5pt}
\newcommand{\XX}{\kern-2pt}
\newcommand{\XXa}{\kern-2.5pt}
\newcommand{\XXX}{\kern-3pt}
\newcommand{\XXXa}{\kern-3.5pt}
\newcommand{\XXXX}{\XXX\X}

\renewcommand{\a}{\kern0.5pt}
\newcommand{\A}{\kern1pt}
\newcommand{\Ax}{\kern1.5pt}
\renewcommand{\AA}{\kern2pt}
\newcommand{\AAx}{\kern2.5pt}
\newcommand{\AAA}{\kern3pt}
\newcommand{\AAAA}{\kern8pt}

% • Fonts
\newcommand{\Ink}{\color{black}}               % black color
\newcommand{\fn}[1]{\textit{\ttfamily{#1}}}    % function        (mono italic gray)
\newcommand{\Type}[1]{\text{\scshape{#1}}\X}  % type            (small-caps)
\newcommand{\builtin}[1]{\XXXX\textsf{ % bold text       (bold-sans-serif)
\fontseries{sbc}\selectfont{#1}}}
\newcommand{\head}[1]{\color{ordGray}          % text            (sans-serif lightest gray)
\textsf{#1}\Ink}

% • Symbols
\newcommand{\Ar}      {\;\Rightarrow\;}          % arrow           (⇒)
\newcommand{\Dt}      {\textbf{.}}               % dot             (.)
\newcommand{\Co}      {\A\mathbb{:}\,}           % colon           (:)
\newcommand{\Sq}      {\a\raisebox{0.5pt}{[\,]}} % square brackets ([])

\newcommand{\num}     {\builtin{number}}
\newcommand{\str}     {\builtin{string}}
\newcommand{\unknown} {\builtin{unknown}}

\newcommand{\Ge}         [1] {\color{sigGray}{\ensuremath{\mathbf{#1}}}\Ink}
\newcommand{\Br}         [1] {\A\text{\textless{#1}\textgreater}}
\newcommand{\Tp}         [1] {\Br{\Ge{#1}}}
\newcommand{\Tpp}        [2] {\Br{\Ge{#1}, \Ge{#2}}}
\newcommand{\TpA}            {\Tp{A}}
\newcommand{\TpB}            {\Tp{B}}
\newcommand{\TpT}            {\Tp{T}}
\newcommand{\TpF}            {\Br{\Ge{F}\ensuremath{\AA\sqsubsetneq\AA}\Type{TypeLambda}}}
\newcommand{\TpFA}           {\Br{\Ge{F}\ensuremath{\AA\sqsubsetneq\AA}\Type{TypeLambda}\A, \Ge{A}}}

\newcommand{\ArbOf}      [1] {\Type{Arbitrary}\Br{#1}}
\newcommand{\ArbsOf}     [1] {\Type{Arbitrary}\Br{#1\Sq}}

\newcommand{\ArbT}       [1] {\ArbOf{\,\Type{#1}\,}}
\newcommand{\ArbP}       [1] {\ArbOf{\Ge{#1}}}

\newcommand{\ArbsT}      [1] {\ArbsOf{\Type{#1}}}
\newcommand{\ArbsP}      [1] {\ArbsOf{\Ge{#1}}}

\newcommand{\ArbNum}         {\ArbOf{\num}}
\newcommand{\ArbStr}         {\ArbOf{\str}}
\newcommand{\ArbUnknown}     {\ArbOf{\unknown}}
\newcommand{\ArbNums}        {\ArbsOf{\num}}

\newcommand{\ArbA}           {\ArbP{A}}
\newcommand{\ArbB}           {\ArbP{B}}
\newcommand{\ArbE}           {\ArbP{E}}
\newcommand{\ArbsA}          {\ArbsP{A}}

\newcommand{\ArbOfA}     [1] {\ArbOf{\A\Type{#1}\A\TpA\XX}}
\newcommand{\EffectArb}  [3] {\ArbOf{\A\Type{Effect}\A\Br{#1#2#3}\XX}}
\newcommand{\EffectArbA}     {\EffectArb{\Ge{A}}{}{}}
\newcommand{\EffectArbE}     {\EffectArb{\ensuremath{\bot},\AA}{Error}{}}
\newcommand{\EffectArbAE}    {\EffectArb{\ensuremath{\Ge{A}},\AA}{Error}{}}

\newcommand{\Either}     [2] {\Type{Either}\Tpp{#1}{#2}}
\newcommand{\Var}        [2] {\textit{#1} \Co \Type{#2}}

\newcommand{\fnab}           {\ensuremath{(\x\Var{a}{\Ge{A}})\XX\Ar\XX\Ge{B}\,}}
\newcommand{\fnaFb}          {\ensuremath{(\x\Var{a}{\Ge{A}})\XX\Ar\XX\Ge{F}\X\Br{\Ge{B}}\XX}}
\newcommand{\fnFab}          {\ensuremath{(\x\Var{a}{\Ge{F}\X\Br{\Ge{A}}})\XX\Ar\XX\Ge{B}}}
